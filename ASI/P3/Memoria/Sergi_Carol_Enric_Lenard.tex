\documentclass[11p]{article}
\usepackage[utf8]{inputenc}
\usepackage{listings}
\usepackage{graphicx}
\title{\huge\textbf{ Memòria pràctica 3:\\ Servei de Correu}}
\author{Sergi Carol\ i Enric Lenard }
\begin{document}
\inputencoding{utf8}
\maketitle
\section{Introducció}

En aquesta pràctica voldrem crear un servei de correu per les nostres maquines virtuals.\\\\
Existeixen varies aplicaccions que permeten realitzar el nostre objectiu, però la que nosaltres utilitzarem serà la anomenada \textbf{exim4} com a MTA i l'aplicacció \textbf{mail} com a MUA.\\\\
En la pràctica crearem tres maquines virtuals noves, conjuntament amb les dues que haviem creat anteriorment, i les configurarem per tal de que funcionin de la següent forma:\\\\

\begin{itemize}
	\item Màquina mail1: Servidor de correu en mode internetconf. Gestiona el correu local i el sortint, accepta ser smarthost de les altres m'aquines i accepta correu entrant dirigit al seu nom i al domini d’autoritat g2.asi.itic.cat. 
	\item Màquina proves: maquina d’usuari en mode smarthostconf. Gestiona el correu local, delega el correu sortint extern a l'smarthost i accepta correu entrant dirigit al seu nom.
	\item Màquina mail3: servidor d’un altre servei en mode smarthostconf. Delega el correu sortint extern a l'smarthost i no accepta correu entrant.
\end{itemize}

Donc començarem a configurar les maquines\\
\newpage
\section{Internetconf}

La màquina configurada com Internetconf actua de la següent forma:

\begin{itemize}
	\item Consulta el registre MX del DNS i lliura els missatges al servidor de correu
	\item Si l'adreça es una adreça local la envia al servidor local corresponent
\end{itemize}

Cal entendre que una maquina configurada com internetconf no és el mateix que si estigues configurada com a smarthost, ja que aquesta última no delaga els missatges als servidors de correus corresponents, sino que els hi envia tot a una altre màquina.\\\\

La configuració ha sigut realitzada utilitzant l'interfície gràfica, però a continuació podem veure la configuració final executant \textbf{debconf-show exim4-config} :\\

\begin{lstlisting}
root@mail1:/# debconf-show exim4-config
  exim4/exim4-config-title:
  exim4/dc_smarthost:
  exim4/no_config: true
* exim4/dc_relay_nets: 172.20.2.7;172.20.2.8
  exim4/dc_postmaster: sergi
* exim4/dc_localdelivery: mbox format in /var/mail/
* exim4/mailname: mail1.g2.asi.itic.cat
* exim4/dc_local_interfaces:
* exim4/dc_relay_domains:
* exim4/dc_minimaldns: false
* exim4/use_split_config: false
* exim4/dc_eximconfig_configtype: internet site; mail is sent and received directly using SMTP
* exim4/dc_other_hostnames: localhost;mail1.g2.asi.itic.cat;g2.asi.itic.cat
  exim4/dc_readhost:
  exim4/hide_mailname:
\end{lstlisting}

Tambè es important modificar l'arxiu \textbf{/etc/exim4/exim4.conf.template} ja que sino no podrem enviar missatges de correu entre diferents grups, ja que per defecte el exim4 bloqueja les IP's privades. En aquest arxiu tenim que canviar el següent:\\

\begin{lstlisting}
dns_lookup

	ignore_target_hosts = 0.0.0.0 :  127.0.0.0/8 : 255.255.255.255
	
	
\end{lstlisting}
A més a més aquesta maquina també farà entrega local aixì que tenim que modicar l'arxiu \textbf{/etc/hosts}.

\begin{lstlisting}
127.0.0.1	localhost
127.0.1.1	mail1.g2.asi.itic.cat
\end{lstlisting}
D'aquesta forma ja tenim recepciò per dominis locals.

\section{Smarthost amb local}

Un cop configurada la primera maquina configurarem la següent, anomenada proves, que serà configurada com a smarthost i podra gestionar el correu local.\\
Com en el cas anterior la configuracio també ha sigut realitzada a traves de un wizard, la configuració la podem veure a continucaió:\\
\begin{lstlisting}
root@proves:/home/sergi# debconf-show exim4-config
* exim4/dc_eximconfig_configtype: mail sent by smarthost; received via SMTP or fetchmail
* exim4/dc_minimaldns: false
* exim4/dc_local_interfaces:
* exim4/dc_relay_nets:
* exim4/dc_other_hostnames: proves.g2.asi.itic.cat
* exim4/dc_localdelivery: mbox format in /var/mail/
  exim4/dc_readhost: proves.g2.asi.itic.cat
* exim4/dc_smarthost: mail1.g2.asi.itic.cat
  exim4/no_config: true
  exim4/dc_postmaster: sergi
* exim4/mailname: proves.g2.asi.itic.cat
* exim4/hide_mailname: false
  exim4/exim4-config-title:
* exim4/use_split_config: false
  exim4/dc_relay_domains:
\end{lstlisting}
També, com en el cas anterior, tenim que modificar l'arxiu \textbf{/etc/hosts}.\\
Aquesta màquina actuarà de la següent forma: En el cas de que vulgui enviar algun correu primer de tot mirarà si es per ell mateix, si ho és el rebrà automaticament. Si el correu no és per a ell l'enviarà automaticament cap a la maquina mail1 (que és el smarthost de proves)

\section{Smarthost sense local}

Aquesta maquina anomenada mail3 és igual que la anterior, però no pot rebre correus, només enviar-ne. La configuració final és la següent:\\
\begin{lstlisting}
root@mail3:/home/sergi# debconf-show exim4-config
* exim4/dc_local_interfaces: 127.0.0.1 ; ::1
  exim4/hide_mailname: true
* exim4/dc_minimaldns: false
* exim4/dc_readhost: mail3.g2.asi.itic.cat
* exim4/use_split_config: false
* exim4/dc_other_hostnames: localhost
* exim4/dc_eximconfig_configtype: mail sent by smarthost; no local mail
* exim4/mailname: mail3.g2.asi.itic.cat
  exim4/no_config: true
  exim4/exim4-config-title:
  exim4/dc_postmaster: sergi
* exim4/dc_smarthost: 172.20.2.6
  exim4/dc_relay_domains:
  exim4/dc_localdelivery: mbox format in /var/mail/
  exim4/dc_relay_nets:
\end{lstlisting}

\section{Modificacció dels registres del DNS}

Per tal de que tot aquest muntatge funcioni tenim que modificar la configuració dels DNS de la pràctica anterior per tal de afegir-hi els registres MX (correu). Per tant modifiquem l'arxiu \textbf{etc/bind/db.g2} per tal d'afegir els registres:

\begin{lstlisting}
g2.asi.itic.cat.        MX 10    mail1.g2.asi.itic.cat.
g2.asi.itic.cat.        MX 10    proves.g2.asi.itic.cat.
g2.asi.itic.cat.        MX 10    mail3.g2.asi.itic.cat.


mail1.g2.asi.itic.cat.  A    172.20.2.6
proves.g2.asi.itic.cat. A    172.20.2.7
mail3.g2.asi.itic.cat.  A    172.20.2.8
\end{lstlisting}
D'aquesta manera fer un \textbf{dig -t mx g2.asi.itic.cat} ens mostra lo següent:\\\\

\centerline{\includegraphics[scale=0.4]{../../../../Imatges/dig_mx.png}}

Podem veure a la secciò de resposta els correus de les nostres màquines.
\newpage

\section{Proves}
\subsection{Enviament intern}

A continuació farem un següit de proves en les quals ens enviarem correus entre les nostres màquines.

\subsubsection{mail1 a proves}

Enviament utilitzant l'eina mail:\\\\
\centerline{\includegraphics[scale=0.4]{../../../../Imatges/mail1_proves.png}}\\\\
Recepciò a proves:\\\\
\centerline{\includegraphics[scale=0.4]{../../../../Imatges/rebuda_proves.png}}
\subsubsection{mail3 a proves}

Aquest cas podrem veure com el missatge passa atraves del smarthost mail1 abans de arribar a proves.\\\\

\centerline{\includegraphics[scale=0.4]{../../../../Imatges/mail3_proves.png}}


\subsection{Enviament a correus externs}

En aquest apartat enviarem correus a maquines externes (en aquest cas seran les de la Silvia i la Nerea).\\\\
Recepció de un correu enviat desde la màquina de la Silvia:\\\\
\centerline{\includegraphics[scale=0.4]{../../../../Imatges/recepcio_silvia.png}}
Com podem veure el destinatari del correu era el domini g2.asi.itic.cat, els correus dirigits a aquest domini aniran cap a la màquina mail1.\\
A continuació un missatge rebut de la màquina de la Nerea:\\\\

\centerline{\includegraphics[scale=0.4]{../../../../Imatges/rebuda_nerea.png}}
Finalment, algunes recepcions de missatges meus que elles han vist:

\centerline{\includegraphics[scale=0.8]{../../../../Imatges/nerea_rebut.png}} 
\centerline{\includegraphics[scale=0.5]{../../../../Imatges/silvia_rebut.jpg}}


\end{document}