\documentclass[11p]{article}
\usepackage[utf8]{inputenc}
\usepackage{listings}
\usepackage{graphicx}
\inputencoding{utf8}

\begin{document}
\part{Xarxa Social}
\section{Introduccio}

\section{Tasques}
\subsection{Obtenir les dades dels usuaris (excepte pwd) que viuen a Manresa}

\begin{lstlisting}

SELECT email,nom,cognom,dataNeixement 
FROM usuaris 
WHERE poblacio=='Manresa';

\end{lstlisting}
Aquesta comanda ens mostra el següent resultat:\\

\begin{tabular}{c c c c}
	enriclenar@gmail.com & Enric & Lenard & 1994 \\
    joanalbets@gmail.com & Joan & Albets & 2001\\
	peregarcia@gmail.com & Pere & Garcia & 1990 \\

\end{tabular}\\\\
\subsection{Obtenir l’email dels usuaris amb cognom ”Albets”}

\begin{lstlisting}

SELECT email 
FROM usuaris 
WHERE cognom=='Albets';

\end{lstlisting}
Executant aquesta comanda obtenim:\\\\
\centerline{joanalbets@gmail.com} \\\\


\subsection{Visualitzar els amics (nom i cognom) de l’usuari ”Pere”, ”Garcia”(estat=Acceptada)}

\begin{lstlisting}
SELECT nom,cognom 
FROM usuaris 
WHERE email IN (SELECT email2 
			FROM amistats 
			WHERE email1='peregarcia@gmail.com' 
			AND estat='Acceptada') 
OR email IN (SELECT email1 
			FROM amistats 
			WHERE email2='peregarcia@gmail.com' 
			AND estat ='Acceptada');

\end{lstlisting}
Amb aquesta comanda busquem dintre la taula d'amistats les persones que han intentat fer amistats amb l'usuari Pere Garcia i també les persones a les quals el Pere Garcia ha intentat fer amistat i la solicitud hagi estat acceptada.\\\\
\begin{tabular}{c c}
	Berta & Capdevila \\
	Enric & Lenard\\
	Anna & Vidella\\
	Sergi & Carol\\
	
\end{tabular}
\\\\
\subsection{Obtenir els amics de l’usaris ”Pere””Garcia”que no són amics de l’usuari ”Jordi””Alba”}
\begin{lstlisting}
SELECT * 
FROM usuaris 
WHERE email IN 
	(SELECT email2 
	FROM amistats 
	WHERE email1='peregarcia@gmail.com' 
	AND estat='Acceptada') 
OR email IN 
	(SELECT email1 
	FROM amistats 
	WHERE email2='peregarcia@gmail.com' 
	AND estat ='Acceptada') 
AND email IN 
	(SELECT email1 
	FROM amistats 
	WHERE email2 = 'jordialba@gmail.com' 
	AND estat != 'Acceptada')
OR email IN 
	(SELECT email2 
	FROM amistats 
	WHERE email1 = 'jordialba@gmail.com' 
	AND estat !='Acceptada')

\end{lstlisting}
Aquesat comanda és similar a la anterior, pero afegim una codicio més en la qual comprovem qui no es amic de l'usuari Jordi Alba\\\\ 
\begin{tabular}{c c c c c c}

bertacapdavila@nose.com & Berta & Capdevila & Navarcles & 1994 & 1234 \\
enriclenard@gmail.com & Enric & Lenard & Manresa & 1994 & Correcalles \\
annavidella@nose.com & Anna & Vidella & Sant Fruitos de Bages & 1991 & 1234 \\

\end{tabular}
\newpage
\subsection{Obtenir el nombre total de peticions d’amistat rebutjades}

\begin{lstlisting}
SELECT count() 
FROM amistats 
WHERE estat == 'Rebutjada';

\end{lstlisting}

Executant aquesta comanda contem el numero de peticions rebutjades.\\\\
\centerline{5}

\subsection{Obtenir les dades (noms,cognoms) d’amics que viuen a Manresa}

\begin{lstlisting}
SELECT nom,cognom FROM usuaris WHERE (poblacio='Manresa') AND ((email IN (SELECT email1 FROM amistats WHERE ((estat='Acceptada') AND (email2 IN (SELECT email FROM usuaris WHERE (poblacio='Manresa')))))) OR (email IN (SELECT email2 FROM amistats WHERE ((estat = 'Acceptada') AND (email1 IN (SELECT email FROM usuaris WHERE (poblacio = 'Manresa')))))));
\end{lstlisting}

Pere|Garcia
Enric|Lenard

\subsection{Obtenir, per cada usuari, el nombre de peticions rebutjades}
\begin{lstlisting}
SELECT nom, count() FROM amistats, usuaris WHERE (estat='Rebutjada' AND email=email1) OR (estat='Rebutjada' AND email=email2) GROUP BY nom;
\end{lstlisting}

Albert|1
Anna|2
Berta|1
Jordi|3
Pere|3
Sergi|1

\subsection{Obtenir els usuaris que no son amics de ”Ana”, ”Vilella”}
\begin{lstlisting}
SELECT nom,cognom FROM usuaris WHERE (email IN (SELECT email1 FROM amistats WHERE email2='annavidella@nose.com' AND estat != 'Acceptada') OR email IN (SELECT email2 FROM amistats WHERE email1='annavidella@nose.com' AND estat !='Acceptada'));
\end{lstlisting}

Jordi|Alba
Sergi|Carol



\part{Empleats}

\section{Introduccio}

\section{Tasques}

\subsection{Obtenir els identificadors i ciutat de residencia dels empleats que treballen per l'empresa "Bank Newton"=12} 
\begin{lstlisting}
SELECT id_empleat,ciutat
FROM empleat 
WHERE (id_empleat IN (
	SELECT id_empleat 
	FROM feina 
	WHERE (id_empresa = 12))
	);
\end{lstlisting}

Resultat\\\\
22|Sant Fruitos de Bages
123|Manresa
1238|Manresa
1239|Manresa\\\\

\subsection{Obtenir totes les dades dels empeleats que treballen per "Bank Newton" guanyen mes de 10000}
\begin{lstlisting}
SELECT id_empleat,carrer,ciutat 
FROM empleat 
WHERE id_empleat IN 
	(SELECT id_empleat 
	FROM feina 
	WHERE id_empresa == 12 
	AND salari >10000
	);
\end{lstlisting}

Resultat:\\\\
1239|Carretera de Cardona|Manresa

\subsection{Obtenir els identificadors dels treballadors que no treballen a ”Bank Newton”}

\begin{lstlisting}
SELECT id_empleat 
FROM empleat 
WHERE id_empleat 
IN (SELECT id_empleat 
	FROM feina 
	WHERE id_empresa != 12
	);
\end{lstlisting}
Resultat:\\\\
1235
1236
1237

\subsection{Trobar tots els treballadors que guanyen m ́es que cada empleat de ”Bank Newton”}

\begin{lstlisting}
SELECT * 
FROM empleat 
WHERE id_empleat IN 
	(SELECT id_empleat 
	FROM feina 
	WHERE id_empresa != 12 
	AND salari > (SELECT sum(salari) 
		FROM feina 
		WHERE id_empresa = 12)
		); 
\end{lstlisting}

Resultat: \\\\
1236|Carretera de Vic|Sant Fruitos de Bages

\subsection{Troba la empresa que t ́e m ́es empleats}

\begin{lstlisting}
SELECT id_empresa,count() AS count 
FROM feina 
GROUP BY id_empresa 
ORDER BY count DESC
LIMIT 1;
\end{lstlisting}
Resultat\\\\
12|4

\subsection{Modifica la ciutat de resid`encia de l’empleat 22 a ’Barcelona’}
\begin{lstlisting}
UPDATE empleat SET ciutat = 'Barcelona' WHERE id_empleat == 22;
SELECT ciutat FROM empleat ;
\end{lstlisting}
Resultat:\\\\

Manresa
Barcelona
Manresa
Sant Fruitos de Bages
Manresa
Manresa
Manresa

\subsection{Apuja el sou de tots els empleats coordinadors un 10}

\begin{lstlisting}
UPDATE feina SET salari = salari + 10  WHERE id_empleat IN (SELECT id_empleat FROM manager);
SELECT id_empleat,salari FROM feina WHERE id_empleat IN (SELECT id_empleat FROM manager);
\end{lstlisting}

Resultat:\\\\

1236|100010
1239|50010

\subsection{Troba el nom de tots els empleats que viuen a la mateixa ciutat on treballen}

\begin{lstlisting}
SELECT id_empleat
FROM feina
WHERE ((SELECT ciutat
FROM empleat) == (SELECT ciutat
FROM empresa));


\end{lstlisting}

\subsection{Troba tots els empleats que viuen a la mateixa ciutat que els seus coordinadors}

\begin{lstlisting}
SELECT id_empleat, id_empleat_coordinador
FROM manager
WHERE (SELECT ciutat
FROM empleat
WHERE id_empleat=id_empleat) = (SELECT ciutat
FROM empleat
WHERE id_empleat=id_empleat_coordinador);

\end{lstlisting}
\end{document}
